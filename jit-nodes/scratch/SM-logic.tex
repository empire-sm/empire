\documentclass[12pt,twoside]{article}
\usepackage{chadstyle}  % Loads my formatting 
%\usepackage{tweaklist}  % I use this package as well; you may need to download

%% Shortcut commands
\newcommand{\proptitle}[1]{\color{ChadBlue} \textnormal{(#1):}}
\newtheorem{proposition}{\color{ChadGreen} Proposition}
\newcommand{\assume}[2]{{\bf{Assumption #1}} (#2)} 
\newcommand{\clr}[1]{{\color{ChadBlue} #1}}
\newcommand{\clrg}[1]{{\color{ChadGreen} #1}}
\newcommand{\Proof}[2]{\newline {\hspace{-\parindent} {\color{ChadGreen}\bf Proof of Proposition}~\ref{#1}.}
{\color{ChadBlue} #2} \vspace{.1in}}

% If you've loaded *tweaklist.sty* above, uncomment these lines:
% Adjust spacing in itemize/enumerate; see tweaklist.sty
%\renewcommand{\enumhook}{\setlength{\topsep}{2pt}%
%  \setlength{\itemsep}{0pt}}
%\renewcommand{\itemhook}{\setlength{\topsep}{2pt}%
%  \setlength{\itemsep}{0pt}}



\begin{document}
\bibliographystyle{aernobold}

%%%%%%%%%%%%%%%%%%%%%%%%%%%%%%%%%%%%%%%%%%%%%%%%%%%%%%%%%%%%%%%%%
% TITLE PAGE
%%%%%%%%%%%%%%%%%%%%%%%%%%%%%%%%%%%%%%%%%%%%%%%%%%%%%%%%%%%%%%%%%

%\begin{singlespacing}
\begin{spacing}{0.9}
\begin{titlepage}

\title{Sonic Multiplicities II: Logic}
\runningheads{Andrew A. Grathwohl}{}

\author{\large \href{http://about.me/agrathwohl}{Andrew A. Grathwohl}\thanks{Dan Stowell, Nick Collins, Paul Davis, sc-users mailing list}\\ {}
   }

\date{\small \today \ -- Version 0.1\\ {\it Preliminary}}
\maketitle
\thispagestyle{empty}

%\clearpage
\vspace{-0.3in}

\begin{abstract}
   Sonic Multiplicities II: Logic
\end{abstract}

\end{titlepage}
\end{spacing}
%\end{singlespacing}


%%%%%%%%%%%%%%%%%%%%%%%%%%%%%%%%%%%%%%%%%%%%%%%%%%%%%%%%%%%
\section{Introduction}
%%%%%%%%%%%%%%%%%%%%%%%%%%%%%%%%%%%%%%%%%%%%%%%%%%%%%%%%%%%

Sometimes, the most important thing to do when trying to solve a problem is to step back and forget all of your formal training. It was this exact practice that I engaged in heavily while writing the Sonic Multiplicities analysis and logic cores. What I set out to do was determine an appropriate means by which to both analyze and judge a performer's choices - in real time, and with a commitment to objectivity. It didn't take long for me to discover that I had vastly overestimated the complexity of this endeavor.

%%%%%%%%%%%%%%%%%%%%%%%%%%%%%%%%%%%%%%%%%%%%%%%%%%%%%%%%%%%
\section{Nonsense}
%%%%%%%%%%%%%%%%%%%%%%%%%%%%%%%%%%%%%%%%%%%%%%%%%%%%%%%%%%%

I began by asking myself, ``what can we do right now?'', and began examining what all sounds emitted from a solo instrumentalist shared. What I learned was that it was very easy to plot all musical decisions from a performer on a simple circumplex chart: x = -100...0...100 (Fluid...Pristine...Percussive); y = -1...1 (Contrasing...Complimenting).

The X axis is the result of analyzing each sound's timbral and pitch quality. A score of -100 (fluid) would suggest a sound that has no transient qualities. A score of 100 (percussive) suggests a sound that is mostly transient in nature, and has a very harshly-sloped release curve.

The Y axis assigns each sound its current ``correctness'' rating. This is essentially a measurement that determines how much each sound contrasts with, or compliments, the overall piece at a particular moment in time.

Each time the analysis engine polls the various outputs in the SM system, it prepares the results as an array and returns a comma-separated table to the Logic Engine:

Sound,X,Y
Input,35,0.70
A,-22,0.62
B,10,0.84
C,32,1.0
D,-45,0.55

The ``Input'' in this example is a violinist. She is bowing with a bit less vibrato than normal, and therefore the sound has a more percussive quality. We know she is bowing with a ``less than normal'' amount of vibrato because we've been examining every sound she's made since her very first note was bowed.


%%%%%%%%%%%%%%%%%%%%%%%%%%%%%%%%%%%%%%%%%%%%%%%%%%%%%%%%%%%
{\small
% Put your bibiliography file here
%\bibliography{Growth.bib}
}

\end{document}
